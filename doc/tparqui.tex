\documentclass[a4paper,14.49998pt]{article}
\usepackage[latin1]{inputenc}
\usepackage{t1enc}
\usepackage[pdftex]{graphicx}
\usepackage[spanish]{babel}
\usepackage{multicol}
\usepackage{fancybox}
\usepackage{fancyhdr}
\usepackage[pdftex,usenames,dvipsnames]{color}
\usepackage{listings}
\usepackage{ucs}
\usepackage{textcomp}
\lstset{
	tabsize=4,
	rulecolor=,
	language=matlab,
        basicstyle=\scriptsize,
        upquote=true,
        aboveskip={1.5\baselineskip},
        columns=fixed,
        showstringspaces=false,
        extendedchars=true,
        breaklines=true,
        prebreak = \raisebox{0ex}[0ex][0ex]{\ensuremath{\hookleftarrow}},
        frame=single,
        showtabs=false,
        showspaces=false,
        showstringspaces=false,
        identifierstyle=\ttfamily,
        keywordstyle=\color[rgb]{0,0,1},
        commentstyle=\color[rgb]{0.133,0.545,0.133},
        stringstyle=\color[rgb]{0.627,0.126,0.941}
}
\pagenumbering{arabic}
\pagestyle{fancy}
\fancyhead[L]{Arqui}
\fancyhead[C]{curso 2011 primer cuatrimestre}
\fancyhead[R]{TP Especial}
\title{\Huge{Arquitectura de las Computadoras\\\vspace{15mm}TP Especial 2011\\\vspace{15mm}Modo protegido con GRUB}}
\begin{document}
\maketitle
\vspace{70mm}
\large{\underline{Autores:}} 
\begin{center}
\begin{tabular}{l r}
\emph{\author{Axel Wassington}} & Legajo: \emph{50124}\\
\emph{\author{Horacio Miguel Gomez}} & Legajo: \emph{50825}\\
\emph{\author{Tom�s Mehdi}} & Legajo: \emph{51014}
\end{tabular}
\end{center}

\pagebreak
\tableofcontents
\pagebreak

\section{Abstract}
El TP Especial consta de un sistema booteable con GRUB instalado, el cual debe
obtener de disco una imagen binaria que correra en Modo protegido. El mismo provee al usuario 
de un shell en el cual puede ejecutar comandos. Similar a Unix/Linux utilizamos la INT80h, 
mediante la cual se pueden ejecutar las instrucciones de assembler IN, OUT y otras.
Al presionar CTRL+ALT+SUPR el sistema se reinicia.

\section{Manual de uso}
Esta seccion esta destinada a proporcionar informaci�n sobre las funcionalidades
del sistema y explicar su modo de uso.
\subsection{Booteo}
El sistema es booteable, por lo que para su inicio solamente se requiere que el CD
se encuentre insertado y la lectora posea la mayor prioridad de booteo. Luego se
mostrar� la pantalla de GRUB. Se puede presionar
enter para proceder, o de lo contrario GRUB continuar � la carga del sistema tras un
n�mero prestablecido de segundos, visible en el ultimo rengl�n de la pantalla. Una vez realizado 
el booteo, se cargar� el shell. A partir de
aqu� se puede comenzar a utilizar los comandos provistos por el shell.
\subsection{Comandos de shell}
La siguiente lista indica la forma de usar a los distintos comandos de el shell.
Es importante destacar que todos las llamadas son case sensitive. En caso
de ingresar un comando que no esta contemplado por el shell se imprime un
mensaje de error y se devuelve el control al usuario.
\subsubsection{cpuspeed}
Muestra la frecuencia de trabajo del CPU, usando la funcion RDTSC(Read Time-Stamp Counter) de 
assembler la cual retorna la cantidad de instrucciones realizada hasta el momento desde el
inicio del procesador y el PIT(Programable Interval Timer). El PIT es un periferico conectado
a la IRQ0 del the master PIC. Utilizando estos elementos podemos obtener una cantidad de
instrucciones en un intervalo de tiempo. La forma de hacerlo es obteniendo un RDTSC, dejar pasar
un tiempo fijo y volver a obtener un RDTSC. El tiempo fijo lo generamos con una cantidad coherente
de timer ticks, para ello no debe ser muy peque�a. Ya teniendo estos datos solo falta hacer 
simples cuentas que nos devolveran la velocidad del CPU en MHz.
\subsubsection{clear}
Limpia la pantalla.
\subsubsection{echo}
Recice una cadena de caracteres luego del nombre del comando e imprime dicha cadena.
\subsubsection{exit}
Cierra el shell.
\pagebreak
\section{INT80h}
Similar a la INT80h de Unix/Linux la INT80h de Arnix segun el valor en el registro EAX elige una instruccion.
Las instrucciones que puede realizar son las siguientes:
\begin{enumerate}
\item Con el valor 3 en EAX hace un read usando los valores de EBX, ECX y EDX. En estos regristros debe
estar el tama�o de lo que se va a leer, el source buffer y un file descriptor(de donde leer) respectivamente.

\item Con el valor 4 en EAX hace un write usando los valores de EBX, ECX y EDX. En estos regristros debe
estar el tama�o de lo que se va a escribir, el source buffer y un file descriptor(donde esribir) 
respectivamente.

\item Con el valor 5 en EAX hace un rdtsc guardando el valor en el registro EBX.
\end{enumerate}
\pagebreak
\section{Codigos fuente}

\subsection{include}
\subitem{defs.h}
\lstinputlisting[language=C]{/home/tmehdi/arnix/include/defs.h}
\subitem{kasm.h}
\lstinputlisting[language=C]{/home/tmehdi/arnix/include/kasm.h}
\subitem{kc.h}
\lstinputlisting[language=C]{/home/tmehdi/arnix/include/kc.h}
\subitem{kernel.h}
\lstinputlisting[language=C]{/home/tmehdi/arnix/include/kernel.h}
\subitem{stdarg.h}
\lstinputlisting[language=C]{/home/tmehdi/arnix/include/stdarg.h}
\subitem{varargs.h}
\lstinputlisting[language=C]{/home/tmehdi/arnix/include/varargs.h}


\subsection{src}

\subsubsection{kernel}
\subitem{kernel.c}
\lstinputlisting[language=C]{/home/tmehdi/arnix/src/kernel.c}
% \subitem{lib.asm}
% \lstinputlisting[language=csh]{/home/tmehdi/arnix/src/libasm.asm}
\subitem{lib.c}
\lstinputlisting[language=C]{/home/tmehdi/arnix/src/libc.c}
\subitem{loader.asm}
\lstinputlisting[language=csh]{/home/tmehdi/arnix/src/loader.asm}
\subsubsection{driver}
% \subitem{keyboard.c}
% \lstinputlisting[language=C]{/home/tmehdi/arnix/src/kernel/driver/keyboard.c}
\subitem{keyboard.h}
\lstinputlisting[language=C]{/home/tmehdi/arnix/src/kernel/driver/keyboard.h}
\subitem{screen.c}
\lstinputlisting[language=C]{/home/tmehdi/arnix/src/kernel/driver/screen.c}
\subitem{screen.h}
\lstinputlisting[language=C]{/home/tmehdi/arnix/src/kernel/driver/screen.h}
\subitem{timer.c}
\lstinputlisting[language=C]{/home/tmehdi/arnix/src/kernel/driver/timer.c}
\subitem{timer.h}
\lstinputlisting[language=C]{/home/tmehdi/arnix/src/kernel/driver/timer.h}
\subsubsection{system}
\subitem{common.h}
\lstinputlisting[language=C]{/home/tmehdi/arnix/src/kernel/system/common.h}
\subitem{idt.c}
\lstinputlisting[language=C]{/home/tmehdi/arnix/src/kernel/system/idt.c}
\subitem{idt.h}
\lstinputlisting[language=C]{/home/tmehdi/arnix/src/kernel/system/idt.h}
\subitem{in\_out.c}
\lstinputlisting[language=C]{/home/tmehdi/arnix/src/kernel/system/in_out.c}
\subitem{in\_out.h}
\lstinputlisting[language=C]{/home/tmehdi/arnix/src/kernel/system/in_out.h}
\subitem{int80.c}
\lstinputlisting[language=C]{/home/tmehdi/arnix/src/kernel/system/int80.c}
\subitem{int80.h}
\lstinputlisting[language=C]{/home/tmehdi/arnix/src/kernel/system/int80.h}
\subitem{isr.c}
\lstinputlisting[language=C]{/home/tmehdi/arnix/src/kernel/system/isr.c}
\subitem{isr.h}
\lstinputlisting[language=C]{/home/tmehdi/arnix/src/kernel/system/isr.h}
\subitem{keyboardlisteners.c}
\lstinputlisting[language=C]{/home/tmehdi/arnix/src/kernel/system/keyboardlisteners.h}
\subitem{keyboardlisteners.h}
\lstinputlisting[language=C]{/home/tmehdi/arnix/src/kernel/system/keyboardlisteners.h}
\subsubsection{asm}
\subitem{idt.asm}
\lstinputlisting[language=csh]{/home/tmehdi/arnix/src/kernel/system/asm/idt.asm}
\subitem{common.asm}
\lstinputlisting[language=csh]{/home/tmehdi/arnix/src/kernel/system/asm/common.asm}



\subsubsection{std}
\subitem{getchar.c}
\lstinputlisting[language=C]{/home/tmehdi/arnix/src/std/getchar.c}
\subitem{printf.c}
\lstinputlisting[language=C]{/home/tmehdi/arnix/src/std/printf.c}
\subitem{scanf.c}
\lstinputlisting[language=C]{/home/tmehdi/arnix/src/std/scanf.c}
\subitem{stdio.h}
\lstinputlisting[language=C]{/home/tmehdi/arnix/src/std/stdio.h}
\subitem{string.c}
\lstinputlisting[language=C]{/home/tmehdi/arnix/src/std/string.c}
\subitem{string.h}
\lstinputlisting[language=C]{/home/tmehdi/arnix/src/std/string.h}
\subitem{systemcall.asm}
\lstinputlisting[language=C]{/home/tmehdi/arnix/src/std/systemcall.asm}
\subitem{systemcall.h}
\lstinputlisting[language=C]{/home/tmehdi/arnix/src/std/systemcall.h}

\subsubsection{user}
\subitem{commands.c}
\lstinputlisting[language=C]{/home/tmehdi/arnix/src/user/commands.c}
\subitem{commands.h}
\lstinputlisting[language=C]{/home/tmehdi/arnix/src/user/commands.h}
\subitem{shell.c}
\lstinputlisting[language=C]{/home/tmehdi/arnix/src/user/shell.c}
\subitem{shell.h}
\lstinputlisting[language=C]{/home/tmehdi/arnix/src/user/shell.h}


\pagebreak

\section{Referencias}
Esta secci�n detalla las distintas fuentes de informaci�n utilizadas para
el desarrollo del TP Especial.
\subsection{Interrupciones}
El manejo de interrupciones es similar al usado en el siguiente tutorial:
\begin{center}
http://www.jamesmolloy.co.uk/tutorial\_html/4.-The\%20GDT\%20and\%20IDT.html\\
\end{center}
Nos pareci� interesante la opci�n de crear un wrapper para las idt y luego desde
C simplemente asignar handlers a las convenientes, un wrapper se encarga de que a 
C le lleguen los parametros.Tambi�n creimos importante tener las entrys para las 
primeras 31 exceptions del procesador, para evitar resets si por ejemplo dividimos por cero.

\subsection{Pantalla}
Nos basamos en la implementaci�n de Linux, la pantalla puede recibir scape chars para 
limpiar la pantalla, o imprimir en colores. Esto era conveniente debido a que al 
utilizar la int80, no necesitamos parametros extra, para estas funcionalidades extra.

\subsection{Reboot}
Luego de probar soluciones sucias, como hacer que el procesador triple-faultee y 
se reinicie la pc. Encontr�mos la solucion de enviar la se�al de reset desde el 
controlador de teclado en:
\begin{center}
http://wiki.osdev.org/Reboot
\end{center}
\subsection{Paginas web utilizadas}
Estos son los puntos de encuentro mas fuertes, esta informaci�n fue informada por mail 
previo a la entrega. En gener�l leimos bastante de:
\begin{center}http://wiki.osdev.org/Main\_Page ( y sus foros )\\
http://www.jamesmolloy.co.uk/tutorial\_html/index.html\\
http://www.osdever.net/tutorials/view/brans-kernel-development-tutorial\\
\end{center}
\end{document}